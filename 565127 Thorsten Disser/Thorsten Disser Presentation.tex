
\documentclass{beamer}
\usepackage{graphicx}
\usepackage[latin1]{inputenc}
\usepackage[T1]{fontenc}
\usepackage[english]{babel}
\usepackage{listings}
\usepackage{xcolor}
\usepackage{eso-pic}
\usepackage{mathrsfs}
\usepackage{url}
\usepackage{amssymb}
\usepackage{amsmath}
\usepackage{multirow}
\usepackage{hyperref}
\usepackage{booktabs}
\usepackage{bbm}
\usepackage{cooltooltips}
\usepackage{colordef}
\usepackage{beamerdefs}
\usepackage{lvblisting}

\pgfdeclareimage[height=2cm]{logobig}{hulogo}

\pgfdeclareimage[height=0.7cm]{logosmall}{Figures/LOB_Logo}

\renewcommand{\titlescale}{1.0}
\renewcommand{\titlescale}{1.0}
\renewcommand{\leftcol}{0.6}


\title[IBT and Copula]{Implied Binomial Tree and Copula Estimation}

\authora{Thorsten Disser} % a-c
\authorb{}
\authorc{}

\def\linka{http://lvb.wiwi.hu-berlin.de}
\def\linkb{}
\def\linkc{}
\institute{Ladislaus von Bortkiewicz Chair of Statistics \\
Humboldt--Universit�t zu Berlin \\}

\hypersetup{pdfpagemode=FullScreen}


\begin{document}

\frame[plain]{
\titlepage
}

\section{Introduction}

\frame{
\frametitle{Outline}

\begin{enumerate}
\item Implied Binomial Tree Derivation
\item Copula Estimation

\end{enumerate}
}

\section{Implied Binomial Tree Derivation}

\frame{
\frametitle{Implied Binomial Tree}
\begin{itemize}
\item Construction adapted to the volatility smile
\item Possibility to price derivative securities
\item Calculation of the state price density (SPD)
\item Calculation of the implied local volatility surface
\end{itemize}
}

\frame{
\frametitle{IBT algorithm}
\begin{itemize}
\item \textcolor{red} {Arrow-Debreu prices} $\lambda_n^i$ (discounted risk-neutral probability) the price of an option that pays 1 in one and only one state i at $n$th level, and otherwise pays 0
\end{itemize}
\begin{equation*}
\begin{aligned}
\lambda_{n+1}^1&=e^{-r\Delta t}\{(1-p_n^1)\lambda_n^1\}\\
\lambda_{n+1}^{i+1}&=e^{-r\Delta t}\{\lambda_n^ip_n^i+\lambda_n^{i+1}(1-p_n^{i+1})\}\\
\lambda_{n+1}^{n+1}&=e^{-r\Delta t}\{\lambda_n^np_n^n\}
\end{aligned}
\end{equation*}
\begin{itemize}
\item \textcolor{red} {Put option price} $P(K, n\Delta t)$
\end{itemize}
\begin{equation*}
\begin{aligned}
P(K, n\Delta t) = \sum_{i=0}^n \lambda_{n+1}^{i+1}\max(K-S_{n+1}^{i+1},0) 
\end{aligned}
\end{equation*}
}

\frame{

\begin{itemize}
\item Define $S_{n+1}^i = S_1^1 = S, i = \frac{n}{2} + 1$, for n even
\item  K = S = $S_n^i$
\item Risk-neutral condition:
\end{itemize}
\begin{equation*}
\begin{aligned}
F_n^i = p_i^nS_{n+1}^{i+1} + (1-p_i^n)S_{n+1}^{i}
\end{aligned}
\end{equation*}
\begin{itemize}
\item Define
\end{itemize}
\begin{equation*}
\begin{aligned}
\rho_u = \sum_{j=i+1}^n \lambda_n^j (F_n^j - S_n^i), \rho_l = \sum_{j=1}^{i-1} \lambda_n^j (S_n^i - F_n^j)
\end{aligned}
\end{equation*}
}

\frame{
\begin{equation*}
\begin{aligned}
P(S,n\Delta t) &=e^{-r\Delta t}[\lambda_n^1(1-p_n^1) \max(S-S_{n+1}^1,0)\\
&+ \sum_{j=1}^{n-1}\{\lambda_n^jp_n^j+\lambda_n^{j+1}(1-p_n^{j+1})\} \max(S-S_{n+1}^{j+1},0)\\
&+\lambda_n^np_n^n \max(S-S_{n+1}^{n+1},0)]\\
&=e^{-r\Delta t}[\lambda_n^1(1-p_n^1)(S-S_{n+1}^1)\\
&+\sum_{j=1}^{i-1}\{\lambda_n^jp_n^j+\lambda_n^{j+1}(1-p_n^{j+1})\}(S-S_{n+1}^{j+1})]
\end{aligned}
\end{equation*}
}

\frame{
\begin{equation*}
\begin{aligned}
P(S,n\Delta t) &=e^{-r\Delta t} \lambda_n^i(1-p_n^i)(S-S_{n+1}^i)\\
&+\sum_{j=1}^{i-1}\lambda_n^j\{(1-p_n^j)(S-S_{n+1}^j)+p_n^j(S-S_{n+1}^{j+1})]\}\\
&=e^{-r\Delta t}\{\lambda_n^{i}(1-p_n^{i})(S-S_{n+1}^i)+\sum_{j=1}^{i-1}\lambda_n^j(S_n^i-F_n^j)\} 
\end{aligned}
\end{equation*}
}

\frame{
\textcolor{red}{Upward}
\begin{equation*}
\begin{aligned}
S_{n+1}^{i+1}&=\frac{S_{n+1}^i\{C(S,n\Delta t)e^{r\Delta t}-\rho_u\}-\lambda_n^iS(F_n^i-S_{n+1}^i)}{C(S,n\Delta t)e^{r\Delta t}-\rho_u-\lambda_n^i(F_n^i-S_{n+1}^i)}
\end{aligned}
\end{equation*}
\textcolor{red}{Downward}
\begin{equation*}
\begin{aligned}
S_{n+1}^i&=\frac{S_{n+1}^{i+1}\{P(S,n\Delta t)e^{r\Delta t}-\rho_l\}-\lambda_n^iS(F_n^i-S_{n+1}^{i+1})}{P(S,n\Delta t)e^{r\Delta t}-\rho_l-\lambda_n^i(F_n^i-S_{n+1}^{i+1})}
\end{aligned}
\end{equation*}
}

\section{Copula Estimation}

\frame{
\frametitle{Copula}
\begin{itemize}
\item Model dependencies between assets in a portfolio
\item Measurement of risk of a portfolio
\item Extrem outcomes are more likely if the assets in a portfolio are highly correlated
\item Reduction of risk by reducing the correlation of the returns 
\item Possibility to model assets with different distributions
\end{itemize}
}

\frame{
\frametitle{Empirical Study}
\begin{itemize}
\item Empirical study using Gaussian, t, Gumbel and Clayton copula\\
\item $S\&P \ 500$ stocks between 2006 and 2015\\
\item Data from three sectors (aeronautic and defense, media and oil exploration) with four companies each
\end{itemize}
}

\frame{
\frametitle{Used Copulae}
Gaussian Copula\\
\begin{equation*}
\begin{aligned}
\textcolor{blue} {C_p^{Gauss}(u_1,...,u_d)=\Phi_p(\Phi^{-1}(u_1),...,\Phi^{-1}(u_d))}
\end{aligned}
\end{equation*}\\
Student's t-copula\\
\begin{equation*}
\begin{aligned}
\textcolor{blue}{{C_{{\nu},{\Psi}}^t}(u_1,..._u_d)={t_{{\nu},{\Psi}}}(t_{\nu}^{-1}(u_1),...,t_{\nu}^{-1}(u_d))}
\end{aligned}
\end{equation*}
}


\frame{
\frametitle{Used Copulae}
Gumbel copula, 1 $\leq \theta \leq \infty$\\
\begin{equation*}
\begin{aligned}
\textcolor{blue} {C_\theta(u_1,...,u_d)=exp[-\{\sum_{i=1}^d(log \ u_i)^\theta\}^{\theta^{-1}}}
\end{aligned}
\end{equation*}
\\
Clayton copula, $0 <\theta$\\
\begin{equation*}
\begin{aligned}
\textcolor{blue} {C_\theta(u_1,...,u_d)=\{(\sum_{i=1}^du_i^{-\theta})-d+1\}^{-\dfrac{1}{\theta}}}
\end{aligned}
\end{equation*}
}

\frame{
\frametitle{Procedure}
\begin{itemize}
\item GARCH for stock return\\
\item Using residuals to estimate dependence structure from copula functions\\
\item Simulate d-dimensional dependence bades on copula\\
\item Portfolio Value-at-Risk estimate 
\end{itemize}
}


\frame{
\frametitle{Companies Included}
Aeronautic and defense
\begin{itemize}
\item Boing, Lockheed Martin, Northrop Grumman, General Dynamics
\end{itemize}
Media
\begin{itemize}
\item CBS, Comcast, Time Warners, Viacom
\end{itemize}
Oil exploration and equipment
\begin{itemize}
\item Devon Energy, Occidental Petroleum, Marathon Oil, ConocoPhillips
\end{itemize}
}

\frame{ 
\frametitle{Tau(Clayton Copula)}
\begin{tabular}{|r|c|c|c|c|}
  \hline
  {} & Boing & LM & NG & GD \\
  \hline
  \hline
  Boing & 1 & 0.06 & 0.52 & 0.43 \\ 
  LM & 0.06 & 1 & 0.05 & 0.01 \\
  NG & 0.52 & 0.05 & 1 & 0.58\\
  GD & 0.43 & 0.01 & 0.58 & 1\\
  \hline
  CBS & 0.31 & 0 & 0.31 & 0.33\\
  Comcast & 0.32 & 0.05 & 0.38 & 0.35\\
  TW & 0.33 & 0.04 & 0.34 & 0.33\\
  Viacom & 0.22 & 0 & 0.21 & 0.22\\
  \hline
  DE & 0.17 & 0 & 0.20 & 0.25\\
  OP & 0.19 & 0 & 0.27 & 0.29\\
  MO & 0.13 & 0 & 0.17 & 0.20\\
  CP & 0.22 & 0 & 0.25 & 0.25\\
  \hline
  \hline
\end{tabular}
}

\frame{ 
\frametitle{Tau(Clayton Copula)}
\begin{tabular}{|r|c|c|c|c|}
  \hline
  {} & CBS & Comcast & TW & Viacom \\
  \hline
  \hline
  Boing & 0.31 & 0.32 & 0.33 & 0.22 \\ 
  LM & 0 & 0.05 & 0.04 & 0 \\
  NG & 0.31 & 0.38 & 0.34 & 0.21\\
  GD & 0.33 & 0.35 & 0.33 & 0.22\\
  \hline
  CBS & 1 & 0.43 & 0.43 & 0.36\\
  Comcast & 0.43 & 1 & 0.45 & 0.31\\
  TW & 0.43 & 0.45 & 1 & 0.37\\
  Viacom & 0.36 & 0.31 & 0.37 & 1\\
  \hline
  DE & 0.30 & 0.25 & 0.21 & 0.19\\
  OP & 0.28 & 0.27 & 0.23 & 0.19\\
  MO & 0.27 & 0.21 & 0.20 & 0.20\\
  CP & 0.29 & 0.24 & 0.23 & 0.19\\
  \hline
  \hline
\end{tabular}
}

\frame{ 
\frametitle{Tau(Clayton Copula)}
\begin{tabular}{|r|c|c|c|c|}
  \hline
  {} & DE & OP & MO & CP \\
  \hline
  \hline
  Boing & 0.17 & 0.19 & 0.13 & 0.22 \\ 
  LM & 0 & 0 & 0 & 0 \\
  NG & 0.20 & 0.27 & 0.17 & 0.25\\
  GD & 0.25 & 0.29 & 0.20 & 0.25\\
  \hline
  CBS & 0.30 & 0.28 & 0.27 & 0.29\\
  Comcast & 0.25 & 0.27 & 0.21 & 0.24\\
  TW & 0.21 & 0.23 & 0.20 & 0.23\\
  Viacom & 0.19 & 0.19 & 0.20 & 0.19\\
  \hline
  DE & 1 & 0.56 & 0.58 & 0.56\\
  OP & 0.56 & 1 & 0.54 & 0.55\\
  MO & 0.58 & 0.54 & 1 & 0.57\\
  CP & 0.56 & 0.55 & 0.57 & 1\\
  \hline
  \hline
\end{tabular}
}

\frame{
\frametitle{Results}
\begin{itemize}
\item Higher dependence of companies in the same sector
\item Similar structure of the other Copulae
\item Gumbel finds in general a higher dependence as the Clayton
\item Gaussian and t-Copula very similar and with higher dependence as the other two
\end{itemize}
}

\end{document}
